\begin{section}
    {Priority Queue \& Heap}
\section*{Priority Queue}

\subsection*{Definition}

Priority queue consists of a set of elements (organized by priority, also called key).

For implement priority queue, there are obvious ways.
\begin{center}
    \begin{tabular}{p{5cm}cc}
        \hline
        & Insert & DeleteMin \\
        \hline
        Normal queue & $O(1)$ & $O(n)$ \\
        Unsorted linked list & $O(1)$ & $O(n)$ \\
        Sorted linked list & $O(n)$ & $O(1)$ \\
        \hline
    \end{tabular}
\end{center}

$O(n)$ seems to much... so we need to find a better way -> Heap!

\subsection*{Heap}

\textbf{heap property}
\begin{itemize}
    \item if B is a child node of A, then $p(A) \le p(B)$.
    \item Implies that an element with the lowest priority is always in the root node (mean-heap) $\leftrightarrow$ (max-heap)
\end{itemize}

To efficiently implenet to priority queue -> Insert and DeleteMin : $O(\log n)$

There are some types of heap : binary, binomial, fibonacci, 2-3, etc.

\subsubsection*{Binary Heap}

Binary Heap satisfying two properties.
(1) Complete binary tree (structural property) (can be implemented in an array), (2) Min tree (Heap order property) (p(node) $\le$ p(children))
\small{We call Complete Binary Tree as CBT.}

\begin{center}
    \includegraphics*[]{img/Meanheap.png}
\end{center}

\begin{itemize}
    \item Insert
    \item DeleteMin
\end{itemize}

\textbf{Change Min Heap to Max Heap}
% // TODO ---

\bigskip
\end{section}