\begin{question}
Determine whether the relation R on the set of all real
numbers is reflexive, symmetric, antisymmetric, and/or
transitive, where $(x, y) \in R$ if and only if
\begin{enumerate}
    \item $x + y = 0$.
    \item $x = \pm y$.
    \item $x-y$ is a rational number.
\end{enumerate}
\end{question}

\par\noindent\rule{\textwidth}{0.5pt}

\subsubsection*{Solutions}

\begin{enumerate}
    \item $x + y = 0$
    \begin{itemize}
        \item \textbf{Reflexive: } $\forall x [x \in A \to (x, x) \in R]$.\\
        Let's assume that $x = 1$, then $1 + 1 = 2 \neq 0$. Thus, it is not reflexive.
        \item \textbf{symmetric: } $\forall x \forall y[(x, y) \in R \to (y, x) \in R]$.\\
        Let's assume that $(x, y) \in R$, then $x + y = 0$, also $y + x = 0$. Thus, it is symmetric.
        \item \textbf{Antisymmetric: } $\forall x \forall y [(x, y) \in R \wedge (y, x) \in R \to x = y]$. \\
        Let's assume that $(x, y) \in R, (y, x) \in R$. then, $x + y = 0, y + x = 0, y = -x$. Thus, it is not antisymmetric.
        \item \textbf{Transitive: } $\forall x \forall y \forall z [(x, y) \in R \wedge (y, z) \in R \to (x, z) \in R]$.\\
        Let's assume that $(x, y) \in R, (y, z) \in R$. then, $x + y = 0, y + z = 0$. However, $x - z = 0$, not $x+z$. So, it is not transitive.
    \end{itemize}
    Relation 1 has: symmetry.
    \item $x = \pm y$
    \begin{enumerate}
        \item \textbf{Reflexive: } $\forall x [x \in A \to (x, x) \in R]$.\\
        For all $x$, $x = \pm x \to x = x$. Thus, it is reflexive.
        \item \textbf{symmetric: } $\forall x \forall y[(x, y) \in R \to (y, x) \in R]$.\\
        Let's assume that $(x, y) \in R$, then $x = \pm y$, also $y = \pm x$. Thus, it is symmetric.
        \item \textbf{Antisymmetric: } $\forall x \forall y [(x, y) \in R \wedge (y, x) \in R \to x = y]$. \\
        Let's assume that $(x, y) \in R, (y, x) \in R$. then, $x = \pm y, y = \pm x$. $x$ and $y$ can be 1 and -1 (not equal). So, it is not antisymmetric.
        \item \textbf{Transitive: } $\forall x \forall y \forall z [(x, y) \in R \wedge (y, z) \in R \to (x, z) \in R]$.\\
        Let's assume that $(x, y) \in R, (y, z) \in R$. then, $x = \pm y, y = \pm z$. This means, $z = \pm (\pm x) = \pm x$. Thus, it is transitive.
    \end{enumerate}
    Relation 2 has: reflexivity, symmetry, transitivity.

    \bigskip\bigskip\bigskip\bigskip\bigskip\bigskip\bigskip
    \begin{flushright}
        \tiny{Problem 1.3 is on the next page}
    \end{flushright}
    \newpage
    \item $x-y$ is a rational number
    \begin{enumerate}
        \item \textbf{Reflexive: } $\forall x [x \in A \to (x, x) \in R]$.\\
        $\forall x \in \mathbb{R}, x - x = 0$ is rational number, so it is reflexive.
        \item \textbf{symmetric: } $\forall x \forall y[(x, y) \in R \to (y, x) \in R]$.\\
        Let's assume that $(x, y) \in R$, then $x - y$ is rational number, then also $-(x - y) = y - x$ is rational number. Thus, it is symmetric.
        \item \textbf{Antisymmetric: } $\forall x \forall y [(x, y) \in R \wedge (y, x) \in R \to x = y]$. \\
        When $x$ is 5 and $y$ is 3, $x - y = 2$ is rational number, $y - x = -2$ is also rational number but $x$ is not equal to $y$. Thus, it is not antisymmetric.
        \item \textbf{Transitive: } $\forall x \forall y \forall z [(x, y) \in R \wedge (y, z) \in R \to (x, z) \in R]$.\\
        Let's assume that $(x, y) \in R, (y, z) \in R$. then, $x - y$ is rational number, $y - z$ is rational number. Thus, $x - z = (x - y) + (y - z)$ is rational number(sum of two rational numbers). So, it is transitive.
    \end{enumerate}
    Relation 3 has: reflexivity, symmetry, transitivity.
\end{enumerate}