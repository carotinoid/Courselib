\newpage
\begin{question}
An algorithm is called optimal for the solution of a problem with respect to a specified operation if there is no algorithm for solving this problem using fewer operations.

\begin{algorithm}
    \caption{Finding the Maximum Element in a Finite Sequence}
    \begin{algorithmic}
    \Procedure{$max$}{$a_1, a_2, \dots, a_n$: integers}
    \State $max := a_1$
    \For{$i := 2$ \textbf{to} $n$}
        \If{$max < a_i$}
            \State {$max := a_i$}
        \EndIf
    \EndFor
    \State \Return $max$ \Comment{$max$ is the largest element}
    \EndProcedure
    \end{algorithmic}
\end{algorithm}

\begin{enumerate}
    \item Show that Algorithm 1 is an optimal algorithm with respect to the number of comparisons of integers.
    \item Is the linear search algorithm optimal with respect to the number of comparisons of integers?
\end{enumerate}
\end{question}

\par\noindent\rule{\textwidth}{0.5pt}

\subsubsection*{Solutions}

\begin{enumerate}
    \item
    Let's assume that there is another algorithm 2 that can find max more efficiently than the given algorithm. However, Algorithm 2 also be required to compare $n-1$ times to find the max in worst case, so it is $O(n)$. Therefore, the given algorithm is optimal with respect to the number of comparisons of integers.
    
    \item
    Let $n$ be the size of the list.
    Then, the linear search algorithm is required to compare $n$ times to find the max in worst case, so it is $O(n)$.
    However, if input is sorted, we can find the max in $O(1)$ time and any value in $O(\log n)$ time using binary search. Therefore, the linear search algorithm is not optimal with respect to the number of comparisons of integers.
\end{enumerate}