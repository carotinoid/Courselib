\newpage
\begin{question}
Suppose that the ciphertext DVE CFMV KF NFEUVI,
REU KYRK ZJ KYV JVVU FW JTZVETV was produced
by encrypting a plaintext message using a shift cipher.
What is the original plaintext?
\end{question}

\par\noindent\rule{\textwidth}{0.5pt}

\subsubsection*{Solutions}
\indent\indent
For decrypting shift cipher, we need to find the key using relative frequences of letters `E' or `T'. In encrypted text ``DVE CFMV KF NFEUVI,
REU KYRK ZJ KYV JVVU FW JTZVETV'', the most frequent letter is `V'. Let's assume that `V' is encypted is `E' in plaintext, then key of decryption function $f^{-1}(x) = (p-k)$ (mod 26) is $k = 17$. So,

\begin{tabular}{|c|c|}
    \hline
    Encrypt & Decrypt \\
    \hline
    A $\rightarrow$ R & A $\rightarrow$ J \\
    B $\rightarrow$ S & B $\rightarrow$ K \\
    C $\rightarrow$ T & C $\rightarrow$ L \\
    D $\rightarrow$ U & D $\rightarrow$ M \\
    E $\rightarrow$ V & E $\rightarrow$ N \\
    F $\rightarrow$ W & F $\rightarrow$ O \\
    G $\rightarrow$ X & G $\rightarrow$ P \\
    H $\rightarrow$ Y & H $\rightarrow$ Q \\
    I $\rightarrow$ Z & I $\rightarrow$ R \\
    J $\rightarrow$ A & J $\rightarrow$ S \\
    $\vdots$ & $\vdots$ \\
    \hline
\end{tabular}

\bigskip
Then, the plain text is,

\textbf{MEN LOVE TO WONDER, AND THAT IS THE SEED OF SCIENCE}.