\begin{center}
\line(1,0){230}
\end{center}

\section*{Chapter 10 \& 11. Number Theory and Cryptography}

\begin{center}
\line(1,0){230}
\end{center}

\subsection*{10.1. Divisibility and Modular Arithmetic}

$a|b$ is read as $a$ divides $b$.

proerties of Divisibility:
\begin{itemize}
    \item If $a|b$ and $b|c$, then $a|c$.
    \item If $a|b$ and $a|c$, then $a|(b+c)$ and $a|(b-c)$.
    \item IF $a|b$, then $a|bc$ for any integer $c$.
\end{itemize}
Division algorithm. $a = dq + r$, d is divisor, q is quotient, r is remainder, a is called dividend. Then, $q = a \text{div} d$ and $r = a \mod d$.

\noindent
Congruence Relation

\noindent
$(\mod m)$ vs $\mod m$

\begin{center}
\line(1,0){230}
\end{center}

\subsection*{10.2. Integer Representations}

representations of integer (n-ary, base b)
base conversion
binary addition and binary multipulication

\begin{center}
\line(1,0){230}
\end{center}

\subsection*{10.3. Primes and Greatest Common Divisors}

prime, fundamental thm of Arithmetic
GCD, finding GCD, euclidean algorithm,
LCM, GCD as linear combination, dividing congruence by an integer

\begin{center}
\line(1,0){230}
\end{center}

\subsection*{10.4. Solving Congruence}

linear congruence, inverse of $a$ modulo $m$,
finding inverse to solve congruence,

\begin{center}
\line(1,0){230}
\end{center}

\subsection*{11.1. Applications of Congruence}

Hashing functions, pseudorandom number, check digits

\begin{center}
\line(1,0){230}
\end{center}

\subsection*{11.2. Cryptography}

Caesar Cipher, shift Cipher
cryptanalysis of shift cipher,
affine cipher, block cipher,