\documentclass[twocolumn]{article}%
\usepackage{amsmath}%
\usepackage{amsfonts}%
\usepackage{amssymb}%
\usepackage{graphicx}
\usepackage{amsthm}
\usepackage{wrapfig}
\usepackage{kotex}
\usepackage{tikz}
%-------------------------------------------
\newtheorem{theorem}{Theorem}
\newtheorem{acknowledgement}[theorem]{Acknowledgement}
\newtheorem{algorithm}[theorem]{Algorithm}
\newtheorem{axiom}[theorem]{Axiom}
\newtheorem{case}[theorem]{Case}
\newtheorem{claim}[theorem]{Claim}
\newtheorem{conclusion}[theorem]{Conclusion}
\newtheorem{condition}[theorem]{Condition}
\newtheorem{conjecture}[theorem]{Conjecture}
\newtheorem{corollary}[theorem]{Corollary}
\newtheorem{criterion}[theorem]{Criterion}
\newtheorem{definition}[theorem]{Definition}
\newtheorem{example}[theorem]{Example}
\newtheorem{exercise}[theorem]{Exercise}
\newtheorem{lemma}[theorem]{Lemma}
\newtheorem{notation}[theorem]{Notation}
\newtheorem{problem}[theorem]{Problem}
\newtheorem{proposition}[theorem]{Proposition}
\newtheorem{remark}[theorem]{Remark}
\newtheorem{solution}[theorem]{Solution}
\newtheorem{summary}[theorem]{Summary}

\theoremstyle{definition}
\newtheorem{question}{Question}

\setlength{\textwidth}{7.0in}
\setlength{\oddsidemargin}{-0.35in}
\setlength{\topmargin}{-0.5in}
\setlength{\textheight}{9.0in}
\setlength{\parindent}{0.3in}
\setlength{\columnsep}{0.6in}
\setlength{\columnseprule}{1pt}
\begin{document}

\begin{center}
    \textbf{\Large{Discrete Math}} \\
\end{center}
\begin{flushright}
    Kim JaeHwan
\end{flushright}

\usetikzlibrary{arrows}
\usetikzlibrary{automata}

\begin{center}
\line(1,0){230}
\end{center}

\section*{Chapter 10 \& 11. Number Theory and Cryptography}

\begin{center}
\line(1,0){230}
\end{center}

\subsection*{10.1. Divisibility and Modular Arithmetic}

$a|b$ is read as $a$ divides $b$.

proerties of Divisibility:
\begin{itemize}
    \item If $a|b$ and $b|c$, then $a|c$.
    \item If $a|b$ and $a|c$, then $a|(b+c)$ and $a|(b-c)$.
    \item IF $a|b$, then $a|bc$ for any integer $c$.
\end{itemize}
Division algorithm. $a = dq + r$, d is divisor, q is quotient, r is remainder, a is called dividend. Then, $q = a \text{div} d$ and $r = a \mod d$.

\noindent
Congruence Relation

\noindent
$(\mod m)$ vs $\mod m$

\begin{center}
\line(1,0){230}
\end{center}

\subsection*{10.2. Integer Representations}

representations of integer (n-ary, base b)
base conversion
binary addition and binary multipulication

\begin{center}
\line(1,0){230}
\end{center}

\subsection*{10.3. Primes and Greatest Common Divisors}

prime, fundamental thm of Arithmetic
GCD, finding GCD, euclidean algorithm,
LCM, GCD as linear combination, dividing congruence by an integer

\begin{center}
\line(1,0){230}
\end{center}

\subsection*{10.4. Solving Congruence}

linear congruence, inverse of $a$ modulo $m$,
finding inverse to solve congruence,

\begin{center}
\line(1,0){230}
\end{center}

\subsection*{11.1. Applications of Congruence}

Hashing functions, pseudorandom number, check digits

\begin{center}
\line(1,0){230}
\end{center}

\subsection*{11.2. Cryptography}

Caesar Cipher, shift Cipher
cryptanalysis of shift cipher,
affine cipher, block cipher,
\newpage

\begin{center}
\line(1,0){230}
\end{center}

\section*{Chapter 12 \& 13. Graph Theory}

\begin{center}
\line(1,0){230}
\end{center}

\subsection*{12.1. Graphs and Graph Models}

Graph definition, remarks, some terminology,
Directed graph,
graph Model: computer networks, others, social netowrks, web graphs, software design.

\begin{center}
\line(1,0){230}
\end{center}

\subsection*{12.2. Graph Terminology and Special Types of Graphs}

Basic terminology : neighbors, neighborhood, degree,
thm1: handshaking thm
thm2: degree sum thm
thm3: in digraph, indeg = outdeg
special type of simple graph : complete, cycles, wheel, n-cubes,
bipartite graph,

\begin{center}
\line(1,0){230}
\end{center}

\subsection*{13.1. Representing Graphs and Graph Isomorphism}

Adjacency List,
Adjacency Matrix,
Incidence Matrix,
isomorphism of graph, algorithm

\begin{center}
\line(1,0){230}
\end{center}

\subsection*{13.2. Connectivity}

path, degrees of seperation, erdos number, bacon numbers, connectedness in undirected graph, connected components, connectedness in directed graph, counting paths between vertices

\begin{center}
\line(1,0){230}
\end{center}

\subsection*{13.3. Euler and Hamilton Paths}

Euler path and circuits, necessary condition, algorithm, application,
hamilton path and circuits, sufficient conditions for hamilton circuit

\newpage
.
\newpage
\begin{center}
\line(1,0){230}
\end{center}

\section*{Chapter 14. Induction and Recursion}

\begin{center}
\line(1,0){230}
\end{center}

\subsection*{14.1. Mathematical Induction}

Principle, important point, validity of induction,
how work, mistaken proof by induction, Guideline for induction proof,

\subsection*{14.2. Strong Induction}

strong induction, compare with Mathematical induction, which will be used, fundamental thm of Arithmetic

\subsection*{14.3. Recursive Definitions and Structural Induction}

recursively defined, fibonacci, recursively defined set and structure, string, Well-formed formula in propositional logic, rooted tree, full binary tree, indution and recursively defined set, structure

\subsection*{14.4. Recursive Algorithms}

recursive algorithm, proving, recursion and iteration, merge sort
\newpage
.
\newpage


\begin{center}
\line(1,0){230}
\end{center}

\section*{Chapter 15. Counting}

\begin{center}
\line(1,0){230}
\end{center}

\subsection*{15.1. The Basics of Counting}

We have to count the number of cases to solve a counting problem. For example, uppercase letter 6 digit password which must contain at least one digit, then how many possible passwords are there? The answer is $36^6 - 26^6$. To solve this problem, we can use three basic principles: the product rule, the sum rule, and the subtraction rule. Note that each cases are independent to use these. 

\begin{center}
\line(1,0){230}
\end{center}

\subsection*{15.2. The Pigeonhole Principle}

\begin{theorem}
    \textbf{The pigeonhole principle: }If $k$ is a possible integer and $k+1$ objects are placed into $k$ boxes, then at least one box contains two or more objects.
\end{theorem}
\begin{proof}
    By contradiction. Suppose none of the $k$ boxes has more than one object, then the total number of objects is at most $k$. This is contradiction.
\end{proof}
\begin{corollary}
    A function $f$ from a set with $k+1$ elements to a set with $k$ elements is not one-to-one by the pigeonhole principle.
\end{corollary}

\begin{theorem}
    \textbf{Generalized pigeonhole principle: } If $N$ objects are placed into $k$ boxes, then there is at least one box containing least $\lceil N/k \rceil$ objects.
\end{theorem}
\begin{proof}
    Same as the pigeonhole principle.
\end{proof}

Note a lots of examples.

\begin{center}
\line(1,0){230}
\end{center}

\subsection*{15.3. Permutations and Combinations}

\begin{definition}
    \textbf{Permutations: } A permutation of a set of distinct objects is an ordered arrangement of these objects, An ordered arrangement of $r$ elements of a set is called an $r$-permutation. The number of $r$-permutations of a set with $n$ elements is denoted by $P(n,r)$.
\end{definition}
\begin{definition}
    \textbf{Combinations: } An $r$-combination of elements of a set is an unordered selection of $r$ elements from the set. An $r$-combination is simply a subset of the set with $r$ elements. Notation is $C(n, r)$ or $\binom n r$
\end{definition}

Some easy theorem and corollary.
\begin{enumerate}
    \item $P(n, r) = n(n-1)(n-2)\cdots(n-r+1)$
    \item If $n$ and $r$ are integers with $1 \le r \le n$, then $P(n, r) = \frac{n!}{(n-r)!}$
    \item $C(n, r) = \frac{n!}{r!(n-r)!}$
    \item $C(n, r) = C(n, n-r)$, when $0 \le r \le n$
\end{enumerate}

\begin{center}
\line(1,0){230}
\end{center}

\subsection*{15.4. Binomial Coefficients}

Binomial expression is $(x+y)^n = \sum_{k=0}^n \binom n k x^{n-k}y^k$. This is called binomail theorem. There is useful corollary.

\begin{corollary}
    $$\sum_{k=0}^n \binom n k = 2^n \quad \text{with} \; n \ge 0$$
\end{corollary}
\begin{theorem}
    \textbf{Pascal's Identity: } If $n$ and $k$ are integers with $n\ge k\ge0$, then $binom {n+1} k = \binom n {k-1} + \binom n k$
\end{theorem}
\begin{proof}
    Proof by combinatorial.
\end{proof}

Pascal's triangle is skipped in this paper.

\begin{center}
\line(1,0){230}
\end{center}

\subsection*{15.5 Generalized Permutations and Combinations}

Easy to understand, already known. Just look lecture notes.


\begin{center}
\line(1,0){230}
\end{center}

\section*{Chapter 16. Probability}

\begin{center}
\line(1,0){230}
\end{center}

\subsection*{16.1. Introduction to Discrete Probability}

Key terms:
\begin{itemize}
    \item \textbf{Experiment: } A procedure that yields one of a given set of possible outcomes.
    \item \textbf{Sample space: } The set of all possible outcomes of an experiment.
    \item \textbf{Event: } A subset of the sample space.
\end{itemize}

\begin{definition}
    \textbf{Probability ( by Pierre-Simon Laplace): } If $S$ is a finite  sample space for an experiment and $E$ is an event, then the probability of $E$ is $P(E) = \frac{|E|}{|S|}$. ($0 \le P(E) \le 1$)
\end{definition}

\begin{itemize}
    \item Complement of $E$: $P(\bar E) = 1 - P(E)$
    \item Union of $E_1$ and $E_2$: $P(E_1 \cup E_2) = P(E_1) + P(E_2) - P(E_1 \cap E_2)$
\end{itemize}

\begin{center}
\line(1,0){230}
\end{center}

\subsection*{16.2. Probability Theory}

\begin{itemize}
    \item Assigning Probability: It assumes that all outcomes are equally likely.
    \item Conditional Probability
    $$P(E|F) = \frac{P(E \cap F)}{P(F)}$$
    \item Independence \\
    The events $E$ and $F$ are independent if and only if $P(E \cap F) = P(E)P(F)$ \\
    Pairwise independence and Mutual independence
    \item Bernoulli Trials and the Binomail Distribution\\
    Suppose an experiment can have only two possible outcomes, each performance of the experiment is called a Bernoulli trial.
    \item Random Variables \\
    A random variable is a function from the sample space of an experiment to the set of real numbers. \\
    \textbf{A random variable is a function. It is not a variable, and it is not random!}
\end{itemize}

\begin{center}
\line(1,0){230}
\end{center}

\subsection*{16.3. Bayes' Theorem}

\begin{theorem}
    Suppose that $E$ and $F$ are events from a sample space $S$ such that $P(E) \neq 0$ and $P(F) \neq 0$. Then: $$P(E|F) = \frac{P(E|F)P(F)}{P(E|F)P(F) + P(E|\bar F)P(\bar F)}$$
\end{theorem}
\begin{proof}
    \begin{align*}
        P(F|E) &= \frac{P(E \cap F)}{P(E)} = \frac{P(E\cap F)}{P(E\cap F) + P(E \cap {\bar F})} \\
        &= \frac{P(E|F)P(F)}{P(E|F)P(F) + P(E|\bar F)P(\bar F)}
    \end{align*}
\end{proof}

\begin{theorem}
    \textbf{Generalized Bayes' Theorem: } Suppose that $E$ is an event from a sample space $S$ and that $F_1, F_2, \cdots F_n$ are \textit{mutually exclusive} events, and assume that $P(E) \neq 0$ for $i = 1, 2, \cdots, n$. Then: $$P(F_i|E) = \frac{P(E|F_i)P(F_i)}{\sum_{j=1}^{n} P(E|F_j)P(F_j)}$$
\end{theorem}

Interpreting Bayes' Theorem:
$$P(A|B) = \frac{P(B|A)P(A)}{P(B)}$$
\begin{itemize}
    \item The event of out interest $A$
    \item The event as an observation $B$
    \item Prior probability $P(A)$, based only on our prior knowledge about A with no observation.
    \item Likelihood $P(B|A)$, the probability of observing $B$ when $A$ happens
    \item Posterior probability $P(A|B)$, the probability of $A$ if we observed $B$.
\end{itemize}

Note lecture note if you need ``A little taste of machine learning'' part.

\begin{center}
\line(1,0){230}
\end{center}

\subsection*{16.4. Expected Value and Variance}

\begin{definition}
    \textbf{Expected Value: } The expected value of a random variable $X(s)$ of the random variable $X(s)$ on the sample sace $S$ is equal to $$E(X) = \sum_{s \in S} P(s) \cdot X(s)$$
\end{definition}


Q. What is the expected value of $n$ mutually independent Bernoulli trials with probability $p$ of success? (np)\\

\begin{itemize}
    \item $E(X_1 + X_2 + \dots + X_n) = E(X_1) + E(X_2) + \dots + E(X_n)$
    \item $E(aX + b) = aE(X) + b$
    \item $E(XY) = E(X)E(Y)$ if $X$ and $Y$ are independent.
\end{itemize}

Note lecture note if you need ``Average-case computational complexity'' and ``Varaince'' parts.


\begin{center}
\line(1,0){230}
\end{center}

\section*{Chapter 17. Relations}

\begin{center}
\line(1,0){230}
\end{center}

\subsection*{17.1. Definition and Properties}

Binary relation.

\begin{enumerate}
    \item \textbf{Reflexive: } $\forall x [x \in A \rightarrow (x, x) \in R]$
    \item \textbf{Symmetric: } $\forall x \forall y [(x, y) \in R \rightarrow (y, x) \in R]$
    \item \textbf{Antisymmetric: } \\
    $\forall x \forall y [(x, y) \in R \wedge (y, x) \in R \rightarrow x = y]$
    \item \textbf{Transitive: } \\
    $\forall x \forall y \forall z [(x, y) \in R \wedge (y, z) \in R \rightarrow (x, z) \in R]$
\end{enumerate}
Combining relations: $R_1 \cup R_2, R_1 \cap R_2, R_1 - R_2$. \\
The composition of relations: $R_1 \circ R_2$. \\
Powers of a relation: $R^1 = R, R^{n+1} = R^n \circ R$. \\

\begin{theorem}
    Relation $R$ is transitive if and only if $R^n \subseteq R$ for all $n \geq 1$.
\end{theorem}

\begin{center}
\line(1,0){230}
\end{center}

\subsection*{17.2. Representing Relations}
\begin{enumerate}
    \item \textbf{Ordered pairs: } $R = \{(a, b), (b, c), (c, a)\}$
    \item \textbf{Matrix: } $R$ is relation from $A$ to $B$, and $A$ has $m$ elements, $B$ has $n$ elements. \\
    $R = \begin{bmatrix}
        1 & 0 & 1 \\
        0 & 1 & 0 \\
        0 & 0 & 1
    \end{bmatrix}$ ex. $m=3, n=3$.
    \begin{enumerate}
        \item Reflexivity: All diagonal elements are 1.
        \item Symmetry: $m_{ij} = 1 \Leftrightarrow m_{ji} = 1$.
        \item Antisymm: $m_{ij} = 0 \vee m_{ji} = 0$ when $i \neq j$.
    \end{enumerate}
    \item \textbf{Directed Graph: } Note an example. \\
    \begin{tabular}{cc}
    $R = \begin{bmatrix}
        0 & 1 & 0 & 1 \\
        0 & 1 & 0 & 1 \\
        1 & 1 & 0 & 0 \\
        0 & 1 & 0 & 0 
    \end{bmatrix}$
    &
    \begin{tikzpicture}[->,>=stealth,shorten >=1pt,auto,node distance=2.5cm, semithick]
    \tikzstyle{every state}=[fill=white,draw=black,text=black]

    \node[state] (A)              {$A$};
    \node[state] (B) [right of=A] {$B$};
    \node[state] (C) [below of=B] {$C$};
    \node[state] (D) [below of=A] {$D$};

    \path (A) edge              node {} (B)
    (A) edge              node {} (D)
    (B) edge [loop above] node {} (B)
    (B) edge [bend right]  node {} (D)
    (C) edge              node {} (A)
    (C) edge              node {} (B)
    (D) edge [bend right] node {} (B);

    \end{tikzpicture}
    \end{tabular}
    \begin{enumerate}
        \item Reflexivity: All nodes have a self-loop.
        \item Symmetry: If there is an edge from $A$ to $B$, there is an edge from $B$ to $A$.
        \item Antisymm: If there is an edge from $A$ to $B$, there is no edge from $B$ to $A$.
        \item Transitivity: (x, y) and (y, z) $\rightarrow$ (x, z).
    \end{enumerate}
\end{enumerate}

\begin{center}
\line(1,0){230}
\end{center}

\subsection*{17.3. Closures}

Let $R$ is a relation on a set $A$. Then, $R$ may or may not have the some properties like reflexivity, symmetry, antisymmetry, and transitivity. Then, $S$ is called the \textbf{closure} of $R$ if $R$ with respect to $P$, if there is a relation $S$ with property $P$ containing $R$ such that $S$ is a subset of every relation with property $P$ containing $R$. In other words, $S$ is the smallest relation with property $P$ containing $R$. \\
\begin{enumerate}
    \item \textbf{Reflexive Closure: } \\
    $R \cup \Delta, \Delta = \{(a, a) | a \in A\}$
    \item \textbf{Symmetric Closure: } \\
    $R \cup R^{-1}, R^{-1} = \{(b, a) | (a, b) \in R\}$
    \item \textbf{Transitive Closure: } \\
    * \textbf{Connectivity relation:} $R^*$ consist of the pairs $(a, b)$ such that there is a path of length at least one from $a$ to $b$. Then, $R^* = \cup _{i=1}^{\infty} R^i$
\end{enumerate}
Here are something.\\
\textbf{path: } if $(a, x_1) \in R, (x_1, x_2) \in R, \cdots, (x_{n-1}, b) \in R$, then $(a, b) \in R^n$, The length of path is $n$.
\begin{theorem}
    There is a path of length $n > 0$ from $a$ to $b$ if and only if $(a, b) \in R^n$.
\end{theorem} 

\noindent
Then, how to show $R^*$ is transitive closure of $R$? \\
1. Show $R^*$ is transitive. \\
2. $R^* \subseteq S$ whenever $S$ is a transitive relation containing $R$. \\
//TODO !!!!!!!!!!!!!

\begin{center}
\line(1,0){230}
\end{center}

\subsection*{17.4. Equivalence Relations}

\begin{definition}
    A relation on a set $A$ is called equivalence relation if it is reflexive, symmetric, and transitive. \\
    Two elements $a$ and $b$ that are related by an equivalence relation are called \textbf{equivalent}, denoted by $a \sim b$.
\end{definition}
\begin{example}
    Let $m$ be an integer with $m > 1$. Show that the relation $R = {(a, b) | a \equiv b \mod m}$ is an equivalence relation on the set of integers. Show that the relation R has reflexive, symmetric, and transitive properties.
\end{example}

\begin{definition}
    \textbf{Equivalence class: } Let $R$ be an equivalence relation on a set $A$. The set of all elements that are related to an element $a$ of $A$ is called the equivalence class of $a$, denoted by $[a]_R$. \\
    $[a]_R = \{s|(a, s) \in R\}$. \\
    If $b \in [a]_R$, then $b$ is called a representative of this equivalence class.
\end{definition}

\begin{theorem}
    Let $R$ be an equivalence relation on a set $A$. Then these statements for element $a$ and $b$ of $A$ are equivalent.
    \begin{enumerate}
        \item $aRb$
        \item $[a] = [b]$
        \item $[a] \cap [b] \neq \phi$
    \end{enumerate}
\end{theorem}

\begin{definition}
    \textbf{Partition: } A partition of a set $A$ is a set of nonempty subsets of $A$ such that every element of $A$ is in exactly one of these subsets. $A_i \neq \phi$, $A_i \cap A_j = \phi$ for $i \neq j$, and $\cup_{i=1}^{\infty} A_i = A$.
\end{definition}

The equivalcne classes form a partition of the set $A$ because they split $A$ into disjoint subsets.

\begin{theorem}
    Let $R$ be an equivalence relation on a set $S$. Then the equivalence classes of $R$ form a partition of $S$. Conversely, ggiven a partition $\{A_i | i \in I \}$ of the set $S$, there is an equivalence relation $R$ that has the sets $A_i, i \in I$, as its equivalence classes.
\end{theorem}
\begin{proof}
    Note lecture note.
\end{proof}

\end{document}