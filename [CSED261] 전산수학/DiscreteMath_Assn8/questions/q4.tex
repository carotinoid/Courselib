\newpage
\begin{question}
Give a combinatorial proof that if $n$ is a positive integer then $\sum_{k=0}^{n}k^{2}\binom{n}{k} = n(n+1)2^{n-2}$. [\textbf{Hint:} Show that both sides count the ways to select a subset of a set of $n$ elements together with two not necessarily distinct elements from this subset. Furthermore, express the righthand side as $n(n-1)2^{n-2} + n2^{n-1}$.]
\end{question}

\par\noindent\rule{\textwidth}{0.5pt}

\subsubsection*{Solution}

Given expression is, $$\sum_{k=0}^{n}k^{2}\binom{n}{k} = n(n+1)2^{n-2} = n(n-1)2^{n-2} + n 2^{n-1}.$$\\
We can think about the situation that there are $n$ candidates for committee members and we want to make a committee of any legal size with a leader and two manager, which are not necessarily distinct (i.e., the leader can be one of the managers).\\
On the one hand, we can divide the situation into two cases.
\begin{enumerate}
    \item Select a leader first and determine the others will be or not be the committee members. (2 ways for each member) Then, the number of ways is, $n \cdot 2^{n-1}$.
    \item Select two managers first and determine the others will be or not be the committee members. Then, the number of ways is, $n(n-1)\cdot 2^{n-2}$.
\end{enumerate}
Because the leader can be one of the managers, the total number of ways is $n(n+1)2^{n-2}$ (sum).\\
On the other hand, by the same fixed size $k$, we can select $k$ committee members and choose a leader or two manager from the committee members.\\
Then, the number of ways is, $$\displaystyle\sum_{k=0}^n k \cdot \binom n k + \sum_{k=0}^{n}k(k-1)\binom{n}{k} = \sum_{k=0}^{n}k^2\binom{n}{k}.$$\\
Since we count the same situation in two different ways, the two expressions are equal. So, we can get the result, $$\sum_{k=0}^{n}k^{2}\binom{n}{k} = n(n+1)2^{n-2}.$$

% https://www.quora.com/How-do-you-find-the-sum-sum-limits_-k-0-n-k-2-binom-n-k