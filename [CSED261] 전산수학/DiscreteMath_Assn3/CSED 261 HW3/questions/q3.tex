\newpage
\begin{question}
Show that if $x$ is a real number, then $\lceil x \rceil - \lfloor x \rfloor = 1$ if $x$ is not an integer and $\lceil x \rceil - \lfloor x \rfloor = 1$ if $x$ is an integer. 
\end{question}

\par\noindent\rule{\textwidth}{0.5pt}

\subsubsection*{Solutions}
The floor function, denoted $f(x) = \lfloor x \rfloor$, is the largest integer less than or equal to $x$.

\noindent
The ceiling function, denoted $f(x) = \lceil x \rceil$, is the smallest integer greater than or equal to $x$.

\bigskip
\noindent
Let $x$ be a real number. If $x$ is an integer, then $\lceil x \rceil = \lfloor x \rfloor = x$. So, $\lceil x \rceil - \lfloor x \rfloor = 0$.

\bigskip
\noindent
Else, if $x$ is not integer, $x$ can be presented as $x = n + \alpha$, where $n$ is an integer and $0 < \alpha < 1$. Then, because of the definition of two function, $\lfloor x \rfloor = n$ and $\lceil x \rceil = n + 1$. So, $\lceil x \rceil - \lfloor x \rfloor = 1$.